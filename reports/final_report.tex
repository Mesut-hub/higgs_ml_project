\documentclass[11pt]{article}

    \usepackage[breakable]{tcolorbox}
    \usepackage{parskip} % Stop auto-indenting (to mimic markdown behaviour)
    

    % Basic figure setup, for now with no caption control since it's done
    % automatically by Pandoc (which extracts ![](path) syntax from Markdown).
    \usepackage{graphicx}
    % Keep aspect ratio if custom image width or height is specified
    \setkeys{Gin}{keepaspectratio}
    % Maintain compatibility with old templates. Remove in nbconvert 6.0
    \let\Oldincludegraphics\includegraphics
    % Ensure that by default, figures have no caption (until we provide a
    % proper Figure object with a Caption API and a way to capture that
    % in the conversion process - todo).
    \usepackage{caption}
    \DeclareCaptionFormat{nocaption}{}
    \captionsetup{format=nocaption,aboveskip=0pt,belowskip=0pt}

    \usepackage{float}
    \floatplacement{figure}{H} % forces figures to be placed at the correct location
    \usepackage{xcolor} % Allow colors to be defined
    \usepackage{enumerate} % Needed for markdown enumerations to work
    \usepackage{geometry} % Used to adjust the document margins
    \usepackage{amsmath} % Equations
    \usepackage{amssymb} % Equations
    \usepackage{textcomp} % defines textquotesingle
    % Hack from http://tex.stackexchange.com/a/47451/13684:
    \AtBeginDocument{%
        \def\PYZsq{\textquotesingle}% Upright quotes in Pygmentized code
    }
    \usepackage{upquote} % Upright quotes for verbatim code
    \usepackage{eurosym} % defines \euro

    \usepackage{iftex}
    \ifPDFTeX
        \usepackage[T1]{fontenc}
        \IfFileExists{alphabeta.sty}{
              \usepackage{alphabeta}
          }{
              \usepackage[mathletters]{ucs}
              \usepackage[utf8x]{inputenc}
          }
    \else
        \usepackage{fontspec}
        \usepackage{unicode-math}
    \fi

    \usepackage{fancyvrb} % verbatim replacement that allows latex
    \usepackage{grffile} % extends the file name processing of package graphics
                         % to support a larger range
    \makeatletter % fix for old versions of grffile with XeLaTeX
    \@ifpackagelater{grffile}{2019/11/01}
    {
      % Do nothing on new versions
    }
    {
      \def\Gread@@xetex#1{%
        \IfFileExists{"\Gin@base".bb}%
        {\Gread@eps{\Gin@base.bb}}%
        {\Gread@@xetex@aux#1}%
      }
    }
    \makeatother
    \usepackage[Export]{adjustbox} % Used to constrain images to a maximum size
    \adjustboxset{max size={0.9\linewidth}{0.9\paperheight}}

    % The hyperref package gives us a pdf with properly built
    % internal navigation ('pdf bookmarks' for the table of contents,
    % internal cross-reference links, web links for URLs, etc.)
    \usepackage{hyperref}
    % The default LaTeX title has an obnoxious amount of whitespace. By default,
    % titling removes some of it. It also provides customization options.
    \usepackage{titling}
    \usepackage{longtable} % longtable support required by pandoc >1.10
    \usepackage{booktabs}  % table support for pandoc > 1.12.2
    \usepackage{array}     % table support for pandoc >= 2.11.3
    \usepackage{calc}      % table minipage width calculation for pandoc >= 2.11.1
    \usepackage[inline]{enumitem} % IRkernel/repr support (it uses the enumerate* environment)
    \usepackage[normalem]{ulem} % ulem is needed to support strikethroughs (\sout)
                                % normalem makes italics be italics, not underlines
    \usepackage{soul}      % strikethrough (\st) support for pandoc >= 3.0.0
    \usepackage{mathrsfs}
    

    
    % Colors for the hyperref package
    \definecolor{urlcolor}{rgb}{0,.145,.698}
    \definecolor{linkcolor}{rgb}{.71,0.21,0.01}
    \definecolor{citecolor}{rgb}{.12,.54,.11}

    % ANSI colors
    \definecolor{ansi-black}{HTML}{3E424D}
    \definecolor{ansi-black-intense}{HTML}{282C36}
    \definecolor{ansi-red}{HTML}{E75C58}
    \definecolor{ansi-red-intense}{HTML}{B22B31}
    \definecolor{ansi-green}{HTML}{00A250}
    \definecolor{ansi-green-intense}{HTML}{007427}
    \definecolor{ansi-yellow}{HTML}{DDB62B}
    \definecolor{ansi-yellow-intense}{HTML}{B27D12}
    \definecolor{ansi-blue}{HTML}{208FFB}
    \definecolor{ansi-blue-intense}{HTML}{0065CA}
    \definecolor{ansi-magenta}{HTML}{D160C4}
    \definecolor{ansi-magenta-intense}{HTML}{A03196}
    \definecolor{ansi-cyan}{HTML}{60C6C8}
    \definecolor{ansi-cyan-intense}{HTML}{258F8F}
    \definecolor{ansi-white}{HTML}{C5C1B4}
    \definecolor{ansi-white-intense}{HTML}{A1A6B2}
    \definecolor{ansi-default-inverse-fg}{HTML}{FFFFFF}
    \definecolor{ansi-default-inverse-bg}{HTML}{000000}

    % common color for the border for error outputs.
    \definecolor{outerrorbackground}{HTML}{FFDFDF}

    % commands and environments needed by pandoc snippets
    % extracted from the output of `pandoc -s`
    \providecommand{\tightlist}{%
      \setlength{\itemsep}{0pt}\setlength{\parskip}{0pt}}
    \DefineVerbatimEnvironment{Highlighting}{Verbatim}{commandchars=\\\{\}}
    % Add ',fontsize=\small' for more characters per line
    \newenvironment{Shaded}{}{}
    \newcommand{\KeywordTok}[1]{\textcolor[rgb]{0.00,0.44,0.13}{\textbf{{#1}}}}
    \newcommand{\DataTypeTok}[1]{\textcolor[rgb]{0.56,0.13,0.00}{{#1}}}
    \newcommand{\DecValTok}[1]{\textcolor[rgb]{0.25,0.63,0.44}{{#1}}}
    \newcommand{\BaseNTok}[1]{\textcolor[rgb]{0.25,0.63,0.44}{{#1}}}
    \newcommand{\FloatTok}[1]{\textcolor[rgb]{0.25,0.63,0.44}{{#1}}}
    \newcommand{\CharTok}[1]{\textcolor[rgb]{0.25,0.44,0.63}{{#1}}}
    \newcommand{\StringTok}[1]{\textcolor[rgb]{0.25,0.44,0.63}{{#1}}}
    \newcommand{\CommentTok}[1]{\textcolor[rgb]{0.38,0.63,0.69}{\textit{{#1}}}}
    \newcommand{\OtherTok}[1]{\textcolor[rgb]{0.00,0.44,0.13}{{#1}}}
    \newcommand{\AlertTok}[1]{\textcolor[rgb]{1.00,0.00,0.00}{\textbf{{#1}}}}
    \newcommand{\FunctionTok}[1]{\textcolor[rgb]{0.02,0.16,0.49}{{#1}}}
    \newcommand{\RegionMarkerTok}[1]{{#1}}
    \newcommand{\ErrorTok}[1]{\textcolor[rgb]{1.00,0.00,0.00}{\textbf{{#1}}}}
    \newcommand{\NormalTok}[1]{{#1}}

    % Additional commands for more recent versions of Pandoc
    \newcommand{\ConstantTok}[1]{\textcolor[rgb]{0.53,0.00,0.00}{{#1}}}
    \newcommand{\SpecialCharTok}[1]{\textcolor[rgb]{0.25,0.44,0.63}{{#1}}}
    \newcommand{\VerbatimStringTok}[1]{\textcolor[rgb]{0.25,0.44,0.63}{{#1}}}
    \newcommand{\SpecialStringTok}[1]{\textcolor[rgb]{0.73,0.40,0.53}{{#1}}}
    \newcommand{\ImportTok}[1]{{#1}}
    \newcommand{\DocumentationTok}[1]{\textcolor[rgb]{0.73,0.13,0.13}{\textit{{#1}}}}
    \newcommand{\AnnotationTok}[1]{\textcolor[rgb]{0.38,0.63,0.69}{\textbf{\textit{{#1}}}}}
    \newcommand{\CommentVarTok}[1]{\textcolor[rgb]{0.38,0.63,0.69}{\textbf{\textit{{#1}}}}}
    \newcommand{\VariableTok}[1]{\textcolor[rgb]{0.10,0.09,0.49}{{#1}}}
    \newcommand{\ControlFlowTok}[1]{\textcolor[rgb]{0.00,0.44,0.13}{\textbf{{#1}}}}
    \newcommand{\OperatorTok}[1]{\textcolor[rgb]{0.40,0.40,0.40}{{#1}}}
    \newcommand{\BuiltInTok}[1]{{#1}}
    \newcommand{\ExtensionTok}[1]{{#1}}
    \newcommand{\PreprocessorTok}[1]{\textcolor[rgb]{0.74,0.48,0.00}{{#1}}}
    \newcommand{\AttributeTok}[1]{\textcolor[rgb]{0.49,0.56,0.16}{{#1}}}
    \newcommand{\InformationTok}[1]{\textcolor[rgb]{0.38,0.63,0.69}{\textbf{\textit{{#1}}}}}
    \newcommand{\WarningTok}[1]{\textcolor[rgb]{0.38,0.63,0.69}{\textbf{\textit{{#1}}}}}
    \makeatletter
    \newsavebox\pandoc@box
    \newcommand*\pandocbounded[1]{%
      \sbox\pandoc@box{#1}%
      % scaling factors for width and height
      \Gscale@div\@tempa\textheight{\dimexpr\ht\pandoc@box+\dp\pandoc@box\relax}%
      \Gscale@div\@tempb\linewidth{\wd\pandoc@box}%
      % select the smaller of both
      \ifdim\@tempb\p@<\@tempa\p@
        \let\@tempa\@tempb
      \fi
      % scaling accordingly (\@tempa < 1)
      \ifdim\@tempa\p@<\p@
        \scalebox{\@tempa}{\usebox\pandoc@box}%
      % scaling not needed, use as it is
      \else
        \usebox{\pandoc@box}%
      \fi
    }
    \makeatother

    % Define a nice break command that doesn't care if a line doesn't already
    % exist.
    \def\br{\hspace*{\fill} \\* }
    % Math Jax compatibility definitions
    \def\gt{>}
    \def\lt{<}
    \let\Oldtex\TeX
    \let\Oldlatex\LaTeX
    \renewcommand{\TeX}{\textrm{\Oldtex}}
    \renewcommand{\LaTeX}{\textrm{\Oldlatex}}
    % Document parameters
    % Document title
    \title{final\_report}
    
    
    
    
    
    
    
% Pygments definitions
\makeatletter
\def\PY@reset{\let\PY@it=\relax \let\PY@bf=\relax%
    \let\PY@ul=\relax \let\PY@tc=\relax%
    \let\PY@bc=\relax \let\PY@ff=\relax}
\def\PY@tok#1{\csname PY@tok@#1\endcsname}
\def\PY@toks#1+{\ifx\relax#1\empty\else%
    \PY@tok{#1}\expandafter\PY@toks\fi}
\def\PY@do#1{\PY@bc{\PY@tc{\PY@ul{%
    \PY@it{\PY@bf{\PY@ff{#1}}}}}}}
\def\PY#1#2{\PY@reset\PY@toks#1+\relax+\PY@do{#2}}

\@namedef{PY@tok@w}{\def\PY@tc##1{\textcolor[rgb]{0.73,0.73,0.73}{##1}}}
\@namedef{PY@tok@c}{\let\PY@it=\textit\def\PY@tc##1{\textcolor[rgb]{0.24,0.48,0.48}{##1}}}
\@namedef{PY@tok@cp}{\def\PY@tc##1{\textcolor[rgb]{0.61,0.40,0.00}{##1}}}
\@namedef{PY@tok@k}{\let\PY@bf=\textbf\def\PY@tc##1{\textcolor[rgb]{0.00,0.50,0.00}{##1}}}
\@namedef{PY@tok@kp}{\def\PY@tc##1{\textcolor[rgb]{0.00,0.50,0.00}{##1}}}
\@namedef{PY@tok@kt}{\def\PY@tc##1{\textcolor[rgb]{0.69,0.00,0.25}{##1}}}
\@namedef{PY@tok@o}{\def\PY@tc##1{\textcolor[rgb]{0.40,0.40,0.40}{##1}}}
\@namedef{PY@tok@ow}{\let\PY@bf=\textbf\def\PY@tc##1{\textcolor[rgb]{0.67,0.13,1.00}{##1}}}
\@namedef{PY@tok@nb}{\def\PY@tc##1{\textcolor[rgb]{0.00,0.50,0.00}{##1}}}
\@namedef{PY@tok@nf}{\def\PY@tc##1{\textcolor[rgb]{0.00,0.00,1.00}{##1}}}
\@namedef{PY@tok@nc}{\let\PY@bf=\textbf\def\PY@tc##1{\textcolor[rgb]{0.00,0.00,1.00}{##1}}}
\@namedef{PY@tok@nn}{\let\PY@bf=\textbf\def\PY@tc##1{\textcolor[rgb]{0.00,0.00,1.00}{##1}}}
\@namedef{PY@tok@ne}{\let\PY@bf=\textbf\def\PY@tc##1{\textcolor[rgb]{0.80,0.25,0.22}{##1}}}
\@namedef{PY@tok@nv}{\def\PY@tc##1{\textcolor[rgb]{0.10,0.09,0.49}{##1}}}
\@namedef{PY@tok@no}{\def\PY@tc##1{\textcolor[rgb]{0.53,0.00,0.00}{##1}}}
\@namedef{PY@tok@nl}{\def\PY@tc##1{\textcolor[rgb]{0.46,0.46,0.00}{##1}}}
\@namedef{PY@tok@ni}{\let\PY@bf=\textbf\def\PY@tc##1{\textcolor[rgb]{0.44,0.44,0.44}{##1}}}
\@namedef{PY@tok@na}{\def\PY@tc##1{\textcolor[rgb]{0.41,0.47,0.13}{##1}}}
\@namedef{PY@tok@nt}{\let\PY@bf=\textbf\def\PY@tc##1{\textcolor[rgb]{0.00,0.50,0.00}{##1}}}
\@namedef{PY@tok@nd}{\def\PY@tc##1{\textcolor[rgb]{0.67,0.13,1.00}{##1}}}
\@namedef{PY@tok@s}{\def\PY@tc##1{\textcolor[rgb]{0.73,0.13,0.13}{##1}}}
\@namedef{PY@tok@sd}{\let\PY@it=\textit\def\PY@tc##1{\textcolor[rgb]{0.73,0.13,0.13}{##1}}}
\@namedef{PY@tok@si}{\let\PY@bf=\textbf\def\PY@tc##1{\textcolor[rgb]{0.64,0.35,0.47}{##1}}}
\@namedef{PY@tok@se}{\let\PY@bf=\textbf\def\PY@tc##1{\textcolor[rgb]{0.67,0.36,0.12}{##1}}}
\@namedef{PY@tok@sr}{\def\PY@tc##1{\textcolor[rgb]{0.64,0.35,0.47}{##1}}}
\@namedef{PY@tok@ss}{\def\PY@tc##1{\textcolor[rgb]{0.10,0.09,0.49}{##1}}}
\@namedef{PY@tok@sx}{\def\PY@tc##1{\textcolor[rgb]{0.00,0.50,0.00}{##1}}}
\@namedef{PY@tok@m}{\def\PY@tc##1{\textcolor[rgb]{0.40,0.40,0.40}{##1}}}
\@namedef{PY@tok@gh}{\let\PY@bf=\textbf\def\PY@tc##1{\textcolor[rgb]{0.00,0.00,0.50}{##1}}}
\@namedef{PY@tok@gu}{\let\PY@bf=\textbf\def\PY@tc##1{\textcolor[rgb]{0.50,0.00,0.50}{##1}}}
\@namedef{PY@tok@gd}{\def\PY@tc##1{\textcolor[rgb]{0.63,0.00,0.00}{##1}}}
\@namedef{PY@tok@gi}{\def\PY@tc##1{\textcolor[rgb]{0.00,0.52,0.00}{##1}}}
\@namedef{PY@tok@gr}{\def\PY@tc##1{\textcolor[rgb]{0.89,0.00,0.00}{##1}}}
\@namedef{PY@tok@ge}{\let\PY@it=\textit}
\@namedef{PY@tok@gs}{\let\PY@bf=\textbf}
\@namedef{PY@tok@ges}{\let\PY@bf=\textbf\let\PY@it=\textit}
\@namedef{PY@tok@gp}{\let\PY@bf=\textbf\def\PY@tc##1{\textcolor[rgb]{0.00,0.00,0.50}{##1}}}
\@namedef{PY@tok@go}{\def\PY@tc##1{\textcolor[rgb]{0.44,0.44,0.44}{##1}}}
\@namedef{PY@tok@gt}{\def\PY@tc##1{\textcolor[rgb]{0.00,0.27,0.87}{##1}}}
\@namedef{PY@tok@err}{\def\PY@bc##1{{\setlength{\fboxsep}{\string -\fboxrule}\fcolorbox[rgb]{1.00,0.00,0.00}{1,1,1}{\strut ##1}}}}
\@namedef{PY@tok@kc}{\let\PY@bf=\textbf\def\PY@tc##1{\textcolor[rgb]{0.00,0.50,0.00}{##1}}}
\@namedef{PY@tok@kd}{\let\PY@bf=\textbf\def\PY@tc##1{\textcolor[rgb]{0.00,0.50,0.00}{##1}}}
\@namedef{PY@tok@kn}{\let\PY@bf=\textbf\def\PY@tc##1{\textcolor[rgb]{0.00,0.50,0.00}{##1}}}
\@namedef{PY@tok@kr}{\let\PY@bf=\textbf\def\PY@tc##1{\textcolor[rgb]{0.00,0.50,0.00}{##1}}}
\@namedef{PY@tok@bp}{\def\PY@tc##1{\textcolor[rgb]{0.00,0.50,0.00}{##1}}}
\@namedef{PY@tok@fm}{\def\PY@tc##1{\textcolor[rgb]{0.00,0.00,1.00}{##1}}}
\@namedef{PY@tok@vc}{\def\PY@tc##1{\textcolor[rgb]{0.10,0.09,0.49}{##1}}}
\@namedef{PY@tok@vg}{\def\PY@tc##1{\textcolor[rgb]{0.10,0.09,0.49}{##1}}}
\@namedef{PY@tok@vi}{\def\PY@tc##1{\textcolor[rgb]{0.10,0.09,0.49}{##1}}}
\@namedef{PY@tok@vm}{\def\PY@tc##1{\textcolor[rgb]{0.10,0.09,0.49}{##1}}}
\@namedef{PY@tok@sa}{\def\PY@tc##1{\textcolor[rgb]{0.73,0.13,0.13}{##1}}}
\@namedef{PY@tok@sb}{\def\PY@tc##1{\textcolor[rgb]{0.73,0.13,0.13}{##1}}}
\@namedef{PY@tok@sc}{\def\PY@tc##1{\textcolor[rgb]{0.73,0.13,0.13}{##1}}}
\@namedef{PY@tok@dl}{\def\PY@tc##1{\textcolor[rgb]{0.73,0.13,0.13}{##1}}}
\@namedef{PY@tok@s2}{\def\PY@tc##1{\textcolor[rgb]{0.73,0.13,0.13}{##1}}}
\@namedef{PY@tok@sh}{\def\PY@tc##1{\textcolor[rgb]{0.73,0.13,0.13}{##1}}}
\@namedef{PY@tok@s1}{\def\PY@tc##1{\textcolor[rgb]{0.73,0.13,0.13}{##1}}}
\@namedef{PY@tok@mb}{\def\PY@tc##1{\textcolor[rgb]{0.40,0.40,0.40}{##1}}}
\@namedef{PY@tok@mf}{\def\PY@tc##1{\textcolor[rgb]{0.40,0.40,0.40}{##1}}}
\@namedef{PY@tok@mh}{\def\PY@tc##1{\textcolor[rgb]{0.40,0.40,0.40}{##1}}}
\@namedef{PY@tok@mi}{\def\PY@tc##1{\textcolor[rgb]{0.40,0.40,0.40}{##1}}}
\@namedef{PY@tok@il}{\def\PY@tc##1{\textcolor[rgb]{0.40,0.40,0.40}{##1}}}
\@namedef{PY@tok@mo}{\def\PY@tc##1{\textcolor[rgb]{0.40,0.40,0.40}{##1}}}
\@namedef{PY@tok@ch}{\let\PY@it=\textit\def\PY@tc##1{\textcolor[rgb]{0.24,0.48,0.48}{##1}}}
\@namedef{PY@tok@cm}{\let\PY@it=\textit\def\PY@tc##1{\textcolor[rgb]{0.24,0.48,0.48}{##1}}}
\@namedef{PY@tok@cpf}{\let\PY@it=\textit\def\PY@tc##1{\textcolor[rgb]{0.24,0.48,0.48}{##1}}}
\@namedef{PY@tok@c1}{\let\PY@it=\textit\def\PY@tc##1{\textcolor[rgb]{0.24,0.48,0.48}{##1}}}
\@namedef{PY@tok@cs}{\let\PY@it=\textit\def\PY@tc##1{\textcolor[rgb]{0.24,0.48,0.48}{##1}}}

\def\PYZbs{\char`\\}
\def\PYZus{\char`\_}
\def\PYZob{\char`\{}
\def\PYZcb{\char`\}}
\def\PYZca{\char`\^}
\def\PYZam{\char`\&}
\def\PYZlt{\char`\<}
\def\PYZgt{\char`\>}
\def\PYZsh{\char`\#}
\def\PYZpc{\char`\%}
\def\PYZdl{\char`\$}
\def\PYZhy{\char`\-}
\def\PYZsq{\char`\'}
\def\PYZdq{\char`\"}
\def\PYZti{\char`\~}
% for compatibility with earlier versions
\def\PYZat{@}
\def\PYZlb{[}
\def\PYZrb{]}
\makeatother


    % For linebreaks inside Verbatim environment from package fancyvrb.
    \makeatletter
        \newbox\Wrappedcontinuationbox
        \newbox\Wrappedvisiblespacebox
        \newcommand*\Wrappedvisiblespace {\textcolor{red}{\textvisiblespace}}
        \newcommand*\Wrappedcontinuationsymbol {\textcolor{red}{\llap{\tiny$\m@th\hookrightarrow$}}}
        \newcommand*\Wrappedcontinuationindent {3ex }
        \newcommand*\Wrappedafterbreak {\kern\Wrappedcontinuationindent\copy\Wrappedcontinuationbox}
        % Take advantage of the already applied Pygments mark-up to insert
        % potential linebreaks for TeX processing.
        %        {, <, #, %, $, ' and ": go to next line.
        %        _, }, ^, &, >, - and ~: stay at end of broken line.
        % Use of \textquotesingle for straight quote.
        \newcommand*\Wrappedbreaksatspecials {%
            \def\PYGZus{\discretionary{\char`\_}{\Wrappedafterbreak}{\char`\_}}%
            \def\PYGZob{\discretionary{}{\Wrappedafterbreak\char`\{}{\char`\{}}%
            \def\PYGZcb{\discretionary{\char`\}}{\Wrappedafterbreak}{\char`\}}}%
            \def\PYGZca{\discretionary{\char`\^}{\Wrappedafterbreak}{\char`\^}}%
            \def\PYGZam{\discretionary{\char`\&}{\Wrappedafterbreak}{\char`\&}}%
            \def\PYGZlt{\discretionary{}{\Wrappedafterbreak\char`\<}{\char`\<}}%
            \def\PYGZgt{\discretionary{\char`\>}{\Wrappedafterbreak}{\char`\>}}%
            \def\PYGZsh{\discretionary{}{\Wrappedafterbreak\char`\#}{\char`\#}}%
            \def\PYGZpc{\discretionary{}{\Wrappedafterbreak\char`\%}{\char`\%}}%
            \def\PYGZdl{\discretionary{}{\Wrappedafterbreak\char`\$}{\char`\$}}%
            \def\PYGZhy{\discretionary{\char`\-}{\Wrappedafterbreak}{\char`\-}}%
            \def\PYGZsq{\discretionary{}{\Wrappedafterbreak\textquotesingle}{\textquotesingle}}%
            \def\PYGZdq{\discretionary{}{\Wrappedafterbreak\char`\"}{\char`\"}}%
            \def\PYGZti{\discretionary{\char`\~}{\Wrappedafterbreak}{\char`\~}}%
        }
        % Some characters . , ; ? ! / are not pygmentized.
        % This macro makes them "active" and they will insert potential linebreaks
        \newcommand*\Wrappedbreaksatpunct {%
            \lccode`\~`\.\lowercase{\def~}{\discretionary{\hbox{\char`\.}}{\Wrappedafterbreak}{\hbox{\char`\.}}}%
            \lccode`\~`\,\lowercase{\def~}{\discretionary{\hbox{\char`\,}}{\Wrappedafterbreak}{\hbox{\char`\,}}}%
            \lccode`\~`\;\lowercase{\def~}{\discretionary{\hbox{\char`\;}}{\Wrappedafterbreak}{\hbox{\char`\;}}}%
            \lccode`\~`\:\lowercase{\def~}{\discretionary{\hbox{\char`\:}}{\Wrappedafterbreak}{\hbox{\char`\:}}}%
            \lccode`\~`\?\lowercase{\def~}{\discretionary{\hbox{\char`\?}}{\Wrappedafterbreak}{\hbox{\char`\?}}}%
            \lccode`\~`\!\lowercase{\def~}{\discretionary{\hbox{\char`\!}}{\Wrappedafterbreak}{\hbox{\char`\!}}}%
            \lccode`\~`\/\lowercase{\def~}{\discretionary{\hbox{\char`\/}}{\Wrappedafterbreak}{\hbox{\char`\/}}}%
            \catcode`\.\active
            \catcode`\,\active
            \catcode`\;\active
            \catcode`\:\active
            \catcode`\?\active
            \catcode`\!\active
            \catcode`\/\active
            \lccode`\~`\~
        }
    \makeatother

    \let\OriginalVerbatim=\Verbatim
    \makeatletter
    \renewcommand{\Verbatim}[1][1]{%
        %\parskip\z@skip
        \sbox\Wrappedcontinuationbox {\Wrappedcontinuationsymbol}%
        \sbox\Wrappedvisiblespacebox {\FV@SetupFont\Wrappedvisiblespace}%
        \def\FancyVerbFormatLine ##1{\hsize\linewidth
            \vtop{\raggedright\hyphenpenalty\z@\exhyphenpenalty\z@
                \doublehyphendemerits\z@\finalhyphendemerits\z@
                \strut ##1\strut}%
        }%
        % If the linebreak is at a space, the latter will be displayed as visible
        % space at end of first line, and a continuation symbol starts next line.
        % Stretch/shrink are however usually zero for typewriter font.
        \def\FV@Space {%
            \nobreak\hskip\z@ plus\fontdimen3\font minus\fontdimen4\font
            \discretionary{\copy\Wrappedvisiblespacebox}{\Wrappedafterbreak}
            {\kern\fontdimen2\font}%
        }%

        % Allow breaks at special characters using \PYG... macros.
        \Wrappedbreaksatspecials
        % Breaks at punctuation characters . , ; ? ! and / need catcode=\active
        \OriginalVerbatim[#1,codes*=\Wrappedbreaksatpunct]%
    }
    \makeatother

    % Exact colors from NB
    \definecolor{incolor}{HTML}{303F9F}
    \definecolor{outcolor}{HTML}{D84315}
    \definecolor{cellborder}{HTML}{CFCFCF}
    \definecolor{cellbackground}{HTML}{F7F7F7}

    % prompt
    \makeatletter
    \newcommand{\boxspacing}{\kern\kvtcb@left@rule\kern\kvtcb@boxsep}
    \makeatother
    \newcommand{\prompt}[4]{
        {\ttfamily\llap{{\color{#2}[#3]:\hspace{3pt}#4}}\vspace{-\baselineskip}}
    }
    

    
    % Prevent overflowing lines due to hard-to-break entities
    \sloppy
    % Setup hyperref package
    \hypersetup{
      breaklinks=true,  % so long urls are correctly broken across lines
      colorlinks=true,
      urlcolor=urlcolor,
      linkcolor=linkcolor,
      citecolor=citecolor,
      }
    % Slightly bigger margins than the latex defaults
    
    \geometry{verbose,tmargin=1in,bmargin=1in,lmargin=1in,rmargin=1in}
    
    

\begin{document}
    
    \maketitle
    
    

    
    \section{higgs-ml-project/notebooks/final\_report.ipynb}\label{higgs-ml-projectnotebooksfinal_report.ipynb}

\{ ``cells'': {[} \{ ``cell\_type'': ``markdown'', ``metadata'': \{\},
``source'': {[} ``\# Machine Learning Pipeline: Feature Selection and
Hyperparameter Optimization\n'', ``\#\# Üsküdar Üniversitesi Fen
Bilimleri Enstitüsü\n'', ``\#\#\# Makine Öğrenmesi Final Ödevi'' {]} \},
\{ ``cell\_type'': ``markdown'', ``metadata'': \{\}, ``source'': {[}
``\#\# 1. Giriş ve Veri Seti Tanıtımı'' {]} \}, \{ ``cell\_type'':
``code'', ``execution\_count'': null, ``metadata'': \{\}, ``outputs'':
{[}{]}, ``source'': {[} ``\# Import libraries\n'', ``import numpy as
np\n'', ``import pandas as pd\n'', ``import matplotlib.pyplot as
plt\n'', ``import seaborn as sns\n'', ``from sklearn.model\_selection
import train\_test\_split\n'', ``from pathlib import Path\n'', ``\n'',
``\# Set global styles\n'', ``plt.style.use(`ggplot')\n'',
``sns.set\_palette(`viridis')\n'',
``pd.set\_option(`display.float\_format', `\{:.4f\}'.format)'' {]} \},
\{ ``cell\_type'': ``code'', ``execution\_count'': null, ``metadata'':
\{\}, ``outputs'': {[}{]}, ``source'': {[} ``\# Load the data\n'',
``data\_path = Path(`../data/higgs\_sample.csv')\n'', ``df =
pd.read\_csv(data\_path, header=None)\n'', ``\n'', ``\# Name columns
according to HIGGS dataset documentation\n'', ``columns =
{[}`class\_label'{]} + {[}f'feature\_\{i\}' for i in range(1,
29){]}\n'', ``df.columns = columns\n'', ``\n'', ``\# Display dataset
info\n'', ``print(f"Dataset shape: \{df.shape\}")\n'',
``print("\textbackslash nFirst 5 rows:")\n'', ``display(df.head())\n'',
``\n'', ``\# Class distribution\n'', ``class\_dist =
df{[}`class\_label'{]}.value\_counts(normalize=True)\n'',
``print("\textbackslash nClass distribution:")\n'',
``display(class\_dist)'' {]} \}, \{ ``cell\_type'': ``markdown'',
``metadata'': \{\}, ``source'': {[} ``\#\# 2. Veri Ön İşleme\n'',
``\#\#\# Aykırı Değer Analizi ve Ölçekleme'' {]} \}, \{ ``cell\_type'':
``code'', ``execution\_count'': null, ``metadata'': \{\}, ``outputs'':
{[}{]}, ``source'': {[} ``from src.preprocessing import
OutlierCapper\n'', ``from sklearn.preprocessing import MinMaxScaler\n'',
``\n'', ``\# Separate features and target\n'', ``X =
df.drop(`class\_label', axis=1)\n'', ``y = df{[}`class\_label'{]}\n'',
``\n'', ``\# Outlier handling\n'', ``capper =
OutlierCapper(factor=1.5)\n'', ``X\_capped =
pd.DataFrame(capper.fit\_transform(X), columns=X.columns)\n'', ``\n'',
``\# Visualize before/after for sample features\n'', ``fig, axes =
plt.subplots(2, 2, figsize=(12, 8))\n'',
``sns.boxplot(data=X{[}{[}`feature\_1', `feature\_2'{]}{]}, ax=axes{[}0,
0{]})\n'', ``axes{[}0, 0{]}.set\_title(`Original Features
(Sample)')\n'', ``\n'', ``sns.boxplot(data=X\_capped{[}{[}`feature\_1',
`feature\_2'{]}{]}, ax=axes{[}0, 1{]})\n'', ``axes{[}0,
1{]}.set\_title(`After Outlier Capping (Sample)')\n'', ``\n'', ``\#
Scaling\n'', ``scaler = MinMaxScaler()\n'', ``X\_scaled =
pd.DataFrame(scaler.fit\_transform(X\_capped), columns=X.columns)\n'',
``\n'', ``\# Visualize scaling results\n'',
``sns.histplot(X\_scaled{[}`feature\_1'{]}, kde=True, ax=axes{[}1,
0{]})\n'', ``axes{[}1, 0{]}.set\_title(`Scaled Feature 1
Distribution')\n'', ``\n'', ``sns.histplot(X\_scaled{[}`feature\_2'{]},
kde=True, ax=axes{[}1, 1{]})\n'', ``axes{[}1, 1{]}.set\_title(`Scaled
Feature 2 Distribution')\n'', ``\n'', ``plt.tight\_layout()\n'',
``plt.savefig(`../outputs/figures/preprocessing\_results.png',
dpi=300)\n'', ``plt.show()'' {]} \}, \{ ``cell\_type'': ``markdown'',
``metadata'': \{\}, ``source'': {[} ``\#\# 3. Özellik Seçimi\n'',
``\#\#\# ANOVA F-Skor ile En Önemli 15 Özelliğin Seçimi'' {]} \}, \{
``cell\_type'': ``code'', ``execution\_count'': null, ``metadata'':
\{\}, ``outputs'': {[}{]}, ``source'': {[} ``from src.feature\_selection
import select\_features\n'', ``from sklearn.feature\_selection import
f\_classif\n'', ``\n'', ``\# Feature selection\n'', ``X\_selected,
selected\_idx = select\_features(X\_scaled.values, y.values,
method=`anova', k=15)\n'', ``\n'', ``\# Get selected feature names\n'',
``selected\_features =
X\_scaled.columns{[}selected\_idx{]}.tolist()\n'', ``print(f"Selected
\{len(selected\_features)\} features:")\n'',
``print(selected\_features)\n'', ``\n'', ``\# Calculate ANOVA F-scores
for all features\n'', ``f\_scores, \_ = f\_classif(X\_scaled, y)\n'',
``feature\_scores = pd.DataFrame(\{\n'', '' `Feature':
X\_scaled.columns,\n``,'' `F\_Score':
f\_scores\n``,''\}).sort\_values(`F\_Score',
ascending=False)\n``,''\n``,''\# Visualize feature
importance\n``,''plt.figure(figsize=(12,
8))\n``,''sns.barplot(x=`F\_Score', y=`Feature',
data=feature\_scores.head(20))\n``,''plt.title(`Top 20 Features by ANOVA
F-Score')\n``,''plt.xlabel(`F-Score')\n``,''plt.ylabel(`Feature')\n``,''plt.tight\_layout()\n``,''plt.savefig(`../outputs/figures/feature\_importance.png',
dpi=300)\n``,''plt.show()'' {]} \}, \{ ``cell\_type'': ``markdown'',
``metadata'': \{\}, ``source'': {[} ``\#\# 4. Modelleme ve
Değerlendirme\n'', ``\#\#\# İç İçe Çapraz Doğrulama (Nested
Cross-Validation) ile Model Performansı'' {]} \}, \{ ``cell\_type'':
``code'', ``execution\_count'': null, ``metadata'': \{\}, ``outputs'':
{[}{]}, ``source'': {[} ``import joblib\n'', ``from src.modeling import
get\_models\n'', ``from src.evaluation import evaluate\_model\n'',
``\n'', ``\# Load pre-trained models and results\n'', ``model\_results =
\{\}\n'', ``models = get\_models()\n'', ``\n'', ``for model\_name in
models.keys():\n'', '' try:\n``,'' \# Load metrics\n``,'' metrics\_path
= f'../outputs/results/\{model\_name\}\_metrics.csv'\n``,'' metrics\_df
= pd.read\_csv(metrics\_path)\n``,'' \n``,'' \# Load best model from
first fold\n``,'' model\_path =
f'../outputs/models/\{model\_name\}\_fold1.pkl'\n``,'' model =
joblib.load(model\_path)\n``,'' \n``,'' model\_results{[}model\_name{]}
= \{\n``,'' `metrics': metrics\_df,\n``,'' `model': model\n``,''
\}\n``,'' print(f"Loaded results for \{model\_name\}")\n``,'' except
FileNotFoundError:\n``,'' print(f"Results not found for
\{model\_name\}")\n``,''\n``,''\# Calculate average
metrics\n``,''performance\_summary = {[}{]}\n``,''for model\_name, data
in model\_results.items():\n``,'' avg\_metrics =
data{[}`metrics'{]}.mean().to\_dict()\n``,'' avg\_metrics{[}`Model'{]} =
model\_name\n``,''
performance\_summary.append(avg\_metrics)\n``,''\n``,''performance\_df =
pd.DataFrame(performance\_summary)\n``,''performance\_df =
performance\_df{[}{[}`Model', `Accuracy', `Precision', `Recall', `F1',
`ROC-AUC'{]}{]}\n``,''print("\textbackslash nAverage Performance
Metrics:")\n``,''display(performance\_df)'' {]} \}, \{ ``cell\_type'':
``markdown'', ``metadata'': \{\}, ``source'': {[} ``\#\# 5. Performans
Karşılaştırması ve ROC Eğrileri'' {]} \}, \{ ``cell\_type'': ``code'',
``execution\_count'': null, ``metadata'': \{\}, ``outputs'': {[}{]},
``source'': {[} ``\# Prepare test set for final evaluation\n'',
``X\_train, X\_test, y\_train, y\_test = train\_test\_split(\n'', ''
X\_scaled{[}selected\_features{]}, y, test\_size=0.2, random\_state=42,
stratify=y\n``,'')\n``,''\n``,''\# Plot ROC curves for all
models\n``,''plt.figure(figsize=(10, 8))\n``,''\n``,''for model\_name,
data in model\_results.items():\n``,'' model = data{[}`model'{]}\n``,''
\n``,'' if hasattr(model, "predict\_proba"):\n``,'' y\_proba =
model.predict\_proba(X\_test){[}:, 1{]}\n``,'' else: \# For SVM without
probability\n``,'' decision = model.decision\_function(X\_test)\n``,''
y\_proba = (decision - decision.min()) / (decision.max() -
decision.min())\n``,'' \n``,'' fpr, tpr, \_ = roc\_curve(y\_test,
y\_proba)\n``,'' roc\_auc = auc(fpr, tpr)\n``,'' \n``,'' plt.plot(fpr,
tpr, lw=2, label=f'\{model\_name\} (AUC =
\{roc\_auc:.3f\})`)\n``,''\n``,''plt.plot({[}0, 1{]}, {[}0, 1{]},
color='gray', lw=1, linestyle=`--')\n``,''plt.xlim({[}0.0,
1.0{]})\n``,''plt.ylim({[}0.0, 1.05{]})\n``,''plt.xlabel(`False Positive
Rate')\n``,''plt.ylabel(`True Positive Rate')\n``,''plt.title(`ROC Curve
Comparison')\n``,''plt.legend(loc="lower
right")\n``,''plt.grid(True)\n``,''plt.savefig(`../outputs/figures/roc\_comparison.png',
dpi=300)\n``,''plt.show()'' {]} \}, \{ ``cell\_type'': ``markdown'',
``metadata'': \{\}, ``source'': {[} ``\#\# 6. Sonuçlar ve Yorum\n'',
``\#\#\# En Başarılı Model Analizi'' {]} \}, \{ ``cell\_type'':
``code'', ``execution\_count'': null, ``metadata'': \{\}, ``outputs'':
{[}{]}, ``source'': {[} ``\# Identify best model\n'',
``best\_model\_name = performance\_df.sort\_values(`ROC-AUC',
ascending=False).iloc{[}0{]}{[}`Model'{]}\n'', ``best\_model =
model\_results{[}best\_model\_name{]}{[}`model'{]}\n'', ``best\_metrics
= performance\_df{[}performance\_df{[}`Model'{]} ==
best\_model\_name{]}.iloc{[}0{]}\n'', ``\n'', ``print(f"Best Performing
Model: \{best\_model\_name\}")\n'', ``print(f"ROC-AUC:
\{best\_metrics{[}`ROC-AUC'{]}:.4f\}")\n'', ``print(f"Accuracy:
\{best\_metrics{[}`Accuracy'{]}:.4f\}")\n'', ``print(f"F1 Score:
\{best\_metrics{[}`F1'{]}:.4f\}")\n'', ``\n'', ``\# Show best
hyperparameters\n'', ``if hasattr(best\_model, `best\_params\_'):\n'',
'' print("\textbackslash nBest Hyperparameters:")\n``,''
display(pd.Series(best\_model.best\_params\_).to\_frame(`Value'))'' {]}
\}, \{ ``cell\_type'': ``markdown'', ``metadata'': \{\}, ``source'': {[}
``\#\# 7. Proje Özeti ve Değerlendirme'' {]} \} {]}, ``metadata'': \{
``kernelspec'': \{ ``display\_name'': ``Python 3 (ipykernel)'',
``language'': ``python'', ``name'': ``python3'' \}, ``language\_info'':
\{ ``codemirror\_mode'': \{ ``name'': ``ipython'', ``version'': 3 \},
``file\_extension'': ``.py'', ``mimetype'': ``text/x-python'', ``name'':
``python'', ``nbconvert\_exporter'': ``python'', ``pygments\_lexer'':
``ipython3'', ``version'': ``3.11.5'' \} \}, ``nbformat'': 4,
``nbformat\_minor'': 4 \}


    % Add a bibliography block to the postdoc
    
    
    
\end{document}
